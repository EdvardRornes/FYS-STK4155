\documentclass[12pt]{article}
\usepackage[a4paper, margin=1in]{geometry}
\usepackage{setspace}
\setstretch{1.5}

\title{Course Feedback Form}
\author{}
\date{}

\begin{document}
	
	\maketitle
	
	Top answers from Isak, bottom answers from Edvard.
	
	\section*{Why did you choose this course?}
	\noindent
	I have always been interested in learning machine learning. I've also considered it something which could be useful if/when applying for jobs.
	\\\\
	Mostly just because its something which will get me a job, but also because I'm relatively interested in ML.
	
	\vspace{2cm}
	
	\section*{What was your programming knowledge before you started?}
	I was fairly confident in python, and somewhat good in c++ before starting this course. This course did not teach me much new in terms of pure programming knowledge.
	\\\\
	Programming knowledge before wasn't anything amazing but do have some background in fortran/C++ and of course Python. I def think it did help with everything related to ML, but outside of that only minor improvements. 
	
	\vspace{2cm}
	
	\section*{How do you judge your own level of knowledge on machine learning before and after this course?}
	I feel like I was a good overview of the subject and understand the key concepts of machine learning. We have been able to implement linear/logisitc regression models and neural networks from scratch, but also with the use of libraries such as keras. 
	\\\\
	Pretty much knew nothing formally about it other than just the classic "neurons firing" so it was quite poor. Now I feel like I would have a much easier time being able to actually use ML for real life tasks, and learning to use the libraries is likely gonna be useful for me in the future. It was not very difficult to get into doing the project.
	
	\vspace{2cm}
	
	\section*{Did the projects and the teaching material allow you to deepen your insights about Machine Learning methods?}
	Yes.
	\\\\
	For sure, however I do think that more mathematical and theoretical background is needed to understand many of the tools we use here better. But as a first course it does its job in introducing and being able to use these methods. But a deeper understanding will go far.
	
	\vspace{2cm}
	
	\section*{Project-based teaching and active learning}
	\noindent
	I liked the project structure of the course.
	\\\\
	Projects are cool but time consuming. I think that if the course is meant to be applications based and not as theoretical then it makes sense to have most of the grade be counted from projects. But it could be worth to try a mix of the two, even if its just a 30\% towards grade exam and there are only 2 projects for example. This would ease the workload slightly whilst still having application be the central theme, but with some greater emphasis on understanding why things work.

	
	\vspace{2cm}
	
	\section*{Usefulness of the weekly exercises}
	\noindent
	The weekly exercises were great for keeping up. 
	\\\\
	Yeah I'd say so, many times we could copy paste large parts of code/derivations and of course having your "hand be held" through some of the more complicated parts makes it easier to start off on your own later, like with the NN's.
	\vspace{2cm}
	
	\section*{Lab sessions and lectures}
	\noindent
	I did most of the study on my own. 
	\\\\
	I am on the 3rd semester of my masters so mostly working on thesis, thus I only attended the first few lectures and no lab sessions.
	
	\vspace{2cm}
	
	\section*{How would you improve this course?}
	\noindent
	ER: Whilst I don't have a great answer for later lectures, something which I think would help a lot is to first introduce Ridge/LASSO with the statistical interpretations. It all seemed completely arbitrary for me before this.
	
	\vspace{2cm}
	
	\section*{Preferred communication channel}
	\noindent
	It was no problem sending an email and getting a reply. Forums such as astro-discourse is are nice. 
	\\\\
	Mail.
	\vspace{2cm}
	
	\section*{Weekly updates}
	\noindent
	Yes. 
	\\\\
	I did not pay too much attention to these. Also I just need to note: Too many notifications, everything doubles up since (at least in my case) we are signed up to getting notification in both FYS-STK3155 and FYS-STK4155. So every time there is something we get 2 emails with the same stuff then 2 canvas notifications. I can obviously just turn these off but still it seems like you should instead have to opt into one, instead of having to opt out of 3 different things.
	
	\vspace{2cm}
	
	\section*{Accessibility of course material}
	\noindent
	Yes, very.
	\\\\
	Yes.
	
	\vspace{2cm}
	
	\section*{Resources and tools}
	\noindent
	Mostly the jupyter notebook, but also lecture (recordings).
	\\\\
	Mostly Jupyter notebook lecture notes.
	
	\vspace{2cm}
	
	\section*{Alternative resources}
	\noindent
	Youtube has a lot of nice material.
	\\\\
	Jupyter notebook lecture notes, chatgpt, online resources, talking to my group partner.
	
	\vspace{2cm}
	
	\section*{Other topics, impressions, or ideas}
	\noindent
	No.
	\\\\
	I just want to note that getting the email about not having more than 10 figures the day before the deadline on the first project is not very nice. At that point people have pretty much already finished their projects and written about every single figure they have. Thus removing a bunch of figures will at best be a massive waste of time, or at worst require massive rewrites depending on how the figures help with the continuity of the project. If you want to impose rules like this then they should be known beforehand, but perhaps this was something that was mentioned in the lectures or somewhere else that we just did not see. If so then that's on us.
	
	\vspace{2cm}
	
\end{document}